% Prime Sieve with Knuth TeX using the 2*3*5*7*11 wheel(480 cogs out of 2310)
% author: jfbu
% date:   2021/07/27
%
% This file defines syntax for a  "Sieve object" in TeX lingua with its
% associated methods: sieve, reset (not implemented), settocount, outtofile.
%
% Usage:
%
% \NewSieve{<name>}{<range>}
%
% **** THE RANGE MUST BE > 11 ****
%
% \Sieve <name>.sieve{} executes the sieve
% %%% \Sieve <name>.reset{} resets all array entries to 0 (not implemented)
%
% after sieve method has been executed one has access to:
%
% \Sieve <name>.settocount{\macro} sets \macro to the number of primes
% \Sieve <name>.outtofile{<filename>} ships out all found primes,
%    one per line to file <filename>.
%    It will also write to stdout the number of primes found
% 
% Algorithm: sieve of Eratosthenes with a 2*3*5*7*11 wheel
%
%
% TeXnicalities:
%
% - ////                                                   \\\\
%   This library file uses only base Knuth's tex, i.e. it
%   doesn't use the e-TeX extensions such as \numexpr but,
%   only the TeX's syntax with \multiply, \divide, \advance.
%   \\\\                                                   ////
%
% - TeX has no native array type so we use "font dimension arrays".
%   There is no way that I know of to release the memory, only
%   to re-use it (and the array size can not be modified as soon
%   as some other font is defined).
%
% - Maximal pdftex font memory is at 147,483,647 32bits words
%   and for sieving up to range N, this needs a bit more than
%   N/2 words of memory.
%
% - luatex has dynamic memory allocation so is not constrained
%   but maximal N is at 999,999,999 due to a limitation
%   (which could be lifted) from a shared_batteries.tex
%   auxiliary macro (square root)


% activate @ as a letter for being more mysterious
\catcode`@ 11
% and use also _ as a letter to appear even more TeXnical
\catcode`_ 11

\input shared_batteries.tex

% additional count registers 
\newcount\indexa
\newcount\indexb 
\newcount\indexc
\newcount\indexd

\newcount\cnta
\newcount\cntb
\newcount\cntc
\newcount\cntd
\newcount\cnte
\newcount\cntf

\newcount\cstA
\newcount\cstB

\newcount\instance

% we need a write stream for the \Sieve <sieveobject>.outtofile{} method
\newwrite\out

\newcount\wheelmodulus
\wheelmodulus=2310 % 2 * 3 * 5 * 7 * 11
% construction of the wheel (it is done once only)
\font\wheelcogids=cmr10 at 666sp
% we will completely ignore even indices
%
% ultimately cogids[n] for 1<= n <= 2310 will be the number
% of integers m<=n prime to 2310. In other terms it is the
% index i of the largest cog value at most equal to n.
%
% the actual cog value as function of index i will be stored
% in the auxiliary array wheelcogs (which will have size 480)
%
% wheelcognext[n] is never used and has no meaning if n is not a cog.
% If n is a cog it gives the next one. The next one after 2309 is 1.
% It turns out that the next one after 1 is 13 and the next one after
% 13 is 17 etc...
%
% wheelcogdeltas[n] is never used if n is not a cog.
% If n is a cog the gives the difference to the next one. Except that 
% the delta for 2309 is +2 and not -2308. Else the wheel would
% not qualify as a true mono-cycle wheel.


% Let's start finding the cogs, i.e. the invertible classes
% modulo 2310. We do this by a mini sieve eliminating multiples
% of 2, 3, 5, 7, and 11. (but 2 will simply be ignored)
%
% the first 7 or 8 (luatex) fontdimen are set but 3, 5, 7 will
% get sieved out soon and 1 will be handled directly later.
%
% regarding even numbers we handle them by cancel culture.
%
% reserve enough space:
\fontdimen\wheelmodulus\wheelcogids=0sp
% sieve out multiples of three up to 2310
\indexa-3
\Replicate{385}{\advance\indexa6 \fontdimen\indexa\wheelcogids=1sp }%
% sieve out multiples of five up to 2310
\indexa-5
\Replicate{231}{\advance\indexa10 \fontdimen\indexa\wheelcogids=1sp }%
% sieve out multiples of seven up to 2310
\indexa-7
\Replicate{165}{\advance\indexa14 \fontdimen\indexa\wheelcogids=1sp }%
% sieve out multiples of eleven up to 2310
\indexa-11
\Replicate{105}{\advance\indexa22 \fontdimen\indexa\wheelcogids=1sp }%

% before giving to the cogids[n] their final values let's initiate
% the cognext and cogdeltas arrays.

\font\wheelcognext=cmr10 at 667sp
\fontdimen\wheelmodulus\wheelcognext=0sp
% we don't care about 1 to 7 or 8 possibly already set values

\font\wheelcogdeltas=cmr10 at 668sp
\fontdimen\wheelmodulus\wheelcogdeltas=0sp

% we now start creating the cogs array. We don't know yet its
% future size (actually we know it will be 480, but TeX does not
% know yet). As long as we don't use "\font" we can append
% to this array.
\font\wheelcogs=cmr10 at 669sp
\fontdimen1\wheelcogids=1sp % 1 is first cog
\fontdimen1\wheelcogs  =1sp % first cog is at 1
\indexa\@ne
\indexb\@ne
\indexc\@ne
\Replicate{1154}{\advance\indexb\tw@
    \ifnum\fontdimen\indexb\wheelcogids=0 % relatively prime to 2310
      \advance\indexc\@ne
      \fontdimen\indexc\wheelcogs    =\indexb sp %
      %
      \fontdimen\indexb\wheelcogids  =\indexc sp %
      \fontdimen\indexa\wheelcognext =\indexb sp %
      %
      \cnta\indexb\advance\cnta-\indexa
%%% main sieving array will use j->2j+1 representation
%%% whereas the cogwheel is represented k->k
%%% so the deltas are divided by 2 here
   \divide\cnta \tw@
%%%
      \fontdimen\indexa\wheelcogdeltas=\cnta sp %
      %
      \indexa\indexb
    \else
      \fontdimen\indexb\wheelcogids  =\indexc sp
    \fi
}%
\fontdimen2309\wheelcognext=1sp %
\fontdimen2309\wheelcogdeltas=1sp % see divide by 2 above

% we have finished our job with constructing the necessary
% and useful wheel data.
% Let's keep memory of the number of relatively prime classes.
\newcount\wheelcogcount
\wheelcogcount\indexc % i.e. 480

% % debugging during code development to check we got it right
% \indexb-\@ne
%   \immediate\write128{class,
%                       id,
%                       nextclass,
%                       delta
%                       }%
% \Replicate{1155}{\advance\indexb\tw@
%   \immediate\write128{\the\indexb,
%                       \the\numexpr\fontdimen\indexb\wheelcogids,
%                       \the\numexpr\fontdimen\indexb\wheelcognext,
%                       \the\numexpr\fontdimen\indexb\wheelcogdeltas
%                       }%
% }%
% \immediate\write128{\the\wheelcogcount, \the\numexpr\fontdimen480\wheelcogs}
%\bye

% INSTANTIATION OF A SIEVING OBJECT
% =================================

\def\Sieve #1.{\csname _svobj.#1\endcsname}%

\def\NewSieve#1{%
  % #1 is the name given by user
  % As it may need expansion, we do this expansion once immediately
  \edef\tmp{{#1}}%
  \expandafter\_NewSieve\tmp
}%

\def\_NewSieve#1#2{%
  % This is the "init" method for a new Sieve instantiation
  %
  % #1 is the name given by user to the sieve object.
  % #2 is its range (included in array if odd)
  %
  % Access to its methods or attributes is done either via this syntax:
  %    \Sieve foo.<method>{}
  % where the {} is mandatory,
  %
  % Or via this syntax (with CamelCase for the method name)
  %     \Sieve<method>{foo}
  % (some methods may use a second argument)
  %
  % attributes:
  % - range
  % - instance (used internally)
  %
  % methods:
  % sieve: execute the sieving of the array
  % reset: clear the array, cancelling all sieving work
  % settocount: compute the number of primes in the given range
  %             and assigns it to macro argument
  %     (the sieve method must have been executed first)
  % outtofile: outputs primes one per line to given filename
  %     (the sieve method must have been executed first)
  %
  \expandafter\def\csname _svobj.#1\endcsname##1##{\csname _svobj.#1.##1\endcsname}%
  % range A
  \cstA#2\relax
  %  (the ##1 is only to gobble the {} from the method calling syntax)
  \expandafter\edef\csname _svobj.#1.range\endcsname##1{\the\cstA}%
  %
  % array size 2J+1<= A
  % (we don't replace A by largest number prime to 2310 and less than A
  %  but we possibly should for some -surely- limited gain)
  \cnta\cstA
  \advance\cnta-\@ne
  \divide\cnta\tw@
  %
  % object instance number (used internally to label class instantiations)
  \advance\instance\@ne
  \expandafter\edef\csname _svobj.#1.instance\endcsname##1{\the\instance}%
  %
  % an array of font dimensions (each a 32bits word)
  \cntb 669 % as 666, 667, 668, 669 are already in use
  \advance\cntb\instance % instance id at least 1 so this will be 670 at least
  % Initialize a font
  \font\_svobjarray=cmr10 at \cntb sp % each class instantiation impacts memory
  % Create the array of its parameters 
  \fontdimen\cnta\_svobjarray=0sp
  % and save its name in the object attributes
  \expandafter\let\csname _svobjarray.#1\endcsname\_svobjarray
  %
  % Array entries with indices 1 to 7 (or 8 with luatex) are not
  % initialized to zero, let's fix this
  \fontdimen1\_svobjarray=0sp 
  \fontdimen2\_svobjarray=0sp
  \fontdimen3\_svobjarray=0sp
  \fontdimen4\_svobjarray=0sp
  \fontdimen5\_svobjarray=0sp
  \fontdimen6\_svobjarray=0sp
  \fontdimen7\_svobjarray=0sp
  \fontdimen8\_svobjarray=0sp % needed for usage with luatex
  %
  % sieve method (the ##1 gobbles the {} of the syntax)
  %
  \expandafter\def\csname _svobj.#1.sieve\endcsname##1{%
      \SieveSieve{#1}%
  }%
  %
  % reset method (the ##1 gobbles the {} of the syntax)
  %
  \expandafter\def\csname _svobj.#1.reset\endcsname##1{%
      \SieveReset{#1}%
  }%
  %
  % settocount method (assigns to the second argument the count of primes)
  %
  \expandafter\def\csname _svobj.#1.settocount\endcsname{%
      \SieveSetToCount{#1}%
  }%
  %
  % outtofile method (outputs to the filename given as second argument the list
  % of primes, one per line)
  %
  \expandafter\def\csname _svobj.#1.outtofile\endcsname{%
      \SieveOutToFile{#1}%
  }%
  % 
}% end of \_NewSieve

%
% sieve method
%
\def\SieveSieve#1{%
    % Set \_svobjarray to the font array
    \expandafter\let\expandafter\_svobjarray\csname _svobjarray.#1\endcsname
    % Set \cstA to be the sieve range. Attention that \_svobj.<foo>.range
    % is defined to gobble a {}...
    \cstA\csname _svobj.#1.range\endcsname{}\relax
    % Set \cstB to its square root
    % nota bene: move this to the "init" and add corresponding attribute to
    % object data?
    \SetToSqrt\cstA\cstB
    %
    % Start the actual wheel sieving
    % 
    % We need to precompute a priori how many elementary steps of the wheel
    % will happen until we have exhausted all potential factors up to B = isqrt(range)
    % 
    \cntb\cstB
        % make sure we never get an even number modulo 2310 as our cogids array indices
        % start at 1 and the array entries are meaningful only for odd indices
        \ifodd\cntb\else\advance\cntb-\@ne\fi
    \cntc\cntb
    % For the sieve range set to 1000000, this division gives zero as 1000<2310
    \divide\cntc\wheelmodulus
    \cntd\cntc
    \multiply\cntd\wheelcogcount
    \multiply\cntc\wheelmodulus
    \advance\cntb-\cntc
    \advance\cntd\fontdimen\cntb\wheelcogids
    %
    % We need to iterate \cntd-1 times to go from
    % 1 to the ultimate N which is prime to 2310 and at most equal to B
    % stepping only along numbers all primes to 2310
    \advance\cntd-\@ne
% debugging
% \immediate\write128{I will do \the\cntd\space iterations}%
    %
    % \indexa will index sieve array (j maps to 2j+1)
    % and will step by entries of the delta array
    % 
    % \indexc will index cogs modulo 2310 (k mapsto k)
    % and will evolve via the entries of the cognext array
    %
    % \cntf will step by one at each iteration and will thus
    % keep information helpful to know for a given factor F
    % exactly how many F*G's will be sieved out
    %
    \indexa\z@
    \indexc\@ne
    \cntf\@ne
    %
    \Replicate{\the\cntd}{%
      % step indexa by suitable delta
      \advance\indexa\fontdimen\indexc\wheelcogdeltas
      % and now update the class modulo wheel modulus
      \indexc\fontdimen\indexc\wheelcognext
      % update the iteration count
      \advance\cntf\@ne
      % do we have a prime?
      \ifnum\fontdimen\indexa\_svobjarray=0 %
        % yes, so let's sieve out its multiples
        % starting with factor*factor
        %
        \cntc \indexa      % will hold the factor
        \advance\cntc\cntc
        \advance\cntc\@ne % \cntc = 2j+1 = odd factor
        %
        \cntb \cntc
        \multiply \cntb \cntb % factor*factor
        %
        \indexb \cntb
        \divide \indexb \tw@ 
        \fontdimen\indexb\_svobjarray=1sp % factor * factor now sieved out
        %
        % we now need to know exactly how many iterations are needed
        % 
        \cnta \cstA % Range
        \divide\cnta\cntc % Range//factor
        % make sure we never get an even number modulo 2310 as our cogids array indices
        % start at 1 and the array entries are meaningful only for odd indices
        \ifodd\cnta\else\advance\cnta-\@ne\fi
        %
        \cntb\cnta
        \cntd\cntb
        \divide\cntb\wheelmodulus
        \cnte\cntb
        \multiply\cntb\wheelmodulus
        \advance\cntd-\cntb % this is modulo modulus
        %
        \multiply\cnte\wheelcogcount
        \advance\cnte\fontdimen\cntd\wheelcogids % get id of maximal cog to
                                % the left
        % get finally how many iterations must be done: cnte - cntf
        \advance\cnte-\cntf
        \indexd\indexc %  indexd will follow modulo modulus the quantity G
                       %  from F * G. Initially G = F which maps to
                       %  indexc
        % indexb will be (F * G - 1)/2, currently G = F
        % let's work with batches of 1000 at a time
        \cntb\cnte
        \divide\cntb\@m % \@m is 1000
        %
        \Replicate{\the\cntb}{%
          \ReplicateM{%
            % step indexb by suitable delta * factor
            \cntd\cntc % factor
            \multiply\cntd\fontdimen\indexd\wheelcogdeltas
            \advance\indexb\cntd
            % and mark this index as sieved out
            \fontdimen\indexb\_svobjarray=1sp % sieved out 
            % update the class of G modulo modulus
            \indexd\fontdimen\indexd\wheelcognext
           }%
         }%
        \multiply\cntb\@m
        \advance\cnte-\cntb % this is what remains to be done
        \Replicate{\the\cnte}{%
            % step indexb by suitable delta * factor
            \cntd\cntc % factor
            \multiply\cntd\fontdimen\indexd\wheelcogdeltas
            \advance\indexb\cntd
            % and mark this index as sieved out
            \fontdimen\indexb\_svobjarray=1sp % sieved out 
            % update the class of G modulo modulus
            \indexd\fontdimen\indexd\wheelcognext
        }%
      \fi
    }%
}% end of \SieveSieve


%
% settocount method
%
\def\SieveSetToCount#1#2{%
  % #1 is the sieve object name
  % #2 is the file name
    % Set \_svobjarray to the font array
    \expandafter\let\expandafter\_svobjarray\csname _svobjarray.#1\endcsname
    % Set \cstA to be the sieve range. Attention that \_svobj.<foo>.range
    % is defined to gobble a {}...
    \cstA\csname _svobj.#1.range\endcsname{}\relax
    %
    % We need to precompute a priori how many elementary steps of the wheel
    % will happen until we have exhausted all potential factors up to B = isqrt(range)
    % 
    \cnta\cstA
        % make sure we never get an even number modulo 2310 as our cogids array indices
        % start at 1 and the array entries are meaningful only for odd indices
        \ifodd\cnta\else\advance\cnta-\@ne\fi
    \cntc\cnta
    \divide\cntc\wheelmodulus
    \cntd\cntc
    \multiply\cntd\wheelcogcount
    \multiply\cntc\wheelmodulus
    \advance\cnta-\cntc % this is A modulo 2310 (A odd so this is odd too)
    \advance\cntd\fontdimen\cnta\wheelcogids
    %
    % We need to iterate \cntd-1 times to go from
    % 1 to the maximal N which is prime to 2310 and at most A
    % stepping along the way only through numbers primes to 2310
    \advance\cntd-\@ne
% debugging
% \immediate\write128{I will do \the\cntd\space iterations}%
    %
    % \indexa will index sieve array (j maps to 2j+1)
    % and will step by entries of the delta array
    % 
    % \indexc will index cogs modulo 2310 (k mapsto k)
    % and will evolve via the entries of the cognext array
    %
    % \cntf will stepped by one each time a prime is found
    %
    \indexa\z@
    \indexc\@ne
    \cntf 5 % 2, 3, 5, 7, 11
    %
    % let's proceed by batches if 1000 at a time
    \cntb\cntd
    \divide\cntb\@m % divide by 1000
    \Replicate{\the\cntb}{%
     \ReplicateM{%
      % step indexa by suitable delta
      \advance\indexa\fontdimen\indexc\wheelcogdeltas
      % and now update the class modulo wheel modulus
      \indexc\fontdimen\indexc\wheelcognext
      % do we have a prime?
      \ifnum\fontdimen\indexa\_svobjarray=0 %
        % we have a prime
        \advance\cntf\@ne
      \fi
     }%
    }%
    % get how many remain to be checked
    \multiply\cntb\@m
    \advance\cntd-\cntb
    % and do it
    \Replicate{\the\cntd}{%
      % step indexa by suitable delta
      \advance\indexa\fontdimen\indexc\wheelcogdeltas
      % and now update the class modulo wheel modulus
      \indexc\fontdimen\indexc\wheelcognext
      % do we have a prime?
      \ifnum\fontdimen\indexa\_svobjarray=0 %
        % we have a prime
        \advance\cntf\@ne
      \fi
    }%
  % set macro #2 to count of primes
  \edef#2{\the\cntf}%
}% END OF \SieveSetToCount

%
% outtofile method
%
\def\SieveOutToFile#1#2{%
  % #1 is the sieve object name
  % #2 is the file name
  % The sieve method of the #1 object must have been executed first
  \immediate\openout\out=#2\relax
  \immediate\write\out{2}%
  \immediate\write\out{3}%
  \immediate\write\out{5}%
  \immediate\write\out{7}%
  \immediate\write\out{11}%
    % Set \_svobjarray to the font array
    \expandafter\let\expandafter\_svobjarray\csname _svobjarray.#1\endcsname
    % Set \cstA to be the sieve range. Attention that \_svobj.<foo>.range
    % is defined to gobble a {}...
    \cstA\csname _svobj.#1.range\endcsname{}\relax
    %
    % We need to precompute a priori how many elementary steps of the wheel
    % will happen until we have exhausted all potential factors up to B = isqrt(range)
    % 
    \cnta\cstA
        % make sure we never get an even number modulo 2310 as our cogids indices
        % start at 1 and the array entries are meaningful only for odd indices
        \ifodd\cnta\else\advance\cnta-\@ne\fi
    \cntc\cnta
    \divide\cntc\wheelmodulus
    \cntd\cntc
    \multiply\cntd\wheelcogcount
    \multiply\cntc\wheelmodulus
    \advance\cnta-\cntc % this is A modulo 2310 (A odd so this is odd too)
    \advance\cntd\fontdimen\cnta\wheelcogids
    %
    % We need to iterate \cntd-1 times to go from
    % 1 to the maximal N which is prime to 2310 and at most A
    % stepping along the way only through numbers primes to 2310
    \advance\cntd-\@ne
% debugging
% \immediate\write128{I will do \the\cntd\space iterations}%
    %
    % \indexa will index sieve array (j maps to 2j+1)
    % and will step by entries of the delta array
    % 
    % \indexc will index cogs modulo 2310 (k mapsto k)
    % and will evolve via the entries of the cognext array
    %
    % \cntf will stepped by one each time a prime is found
    %
    \indexa\z@
    \indexc\@ne
    \cntf 5 % 2, 3, 5, 7, 11
    %
    % let's proceed by batches if 1000 at a time
    \cntb\cntd
    \divide\cntb\@m % divide by 1000
    \Replicate{\the\cntb}{%
     \ReplicateM{%
      % step indexa by suitable delta
      \advance\indexa\fontdimen\indexc\wheelcogdeltas
      % and now update the class modulo wheel modulus
      \indexc\fontdimen\indexc\wheelcognext
      % do we have a prime?
      \ifnum\fontdimen\indexa\_svobjarray=0 %
        % we have a prime
        \cnta\indexa
        \advance\cnta\cnta
        \advance\cnta\@ne
        \immediate\write\out{\the\cnta}%
        \advance\cntf\@ne
      \fi
     }%
    }%
    % get how many remain to be checked
    \multiply\cntb\@m
    \advance\cntd-\cntb
    % and do it
    \Replicate{\the\cntd}{%
      % step indexa by suitable delta
      \advance\indexa\fontdimen\indexc\wheelcogdeltas
      % and now update the class modulo wheel modulus
      \indexc\fontdimen\indexc\wheelcognext
      % do we have a prime?
      \ifnum\fontdimen\indexa\_svobjarray=0 %
      % we have a prime
        \cnta\indexa
        \advance\cnta\cnta
        \advance\cnta\@ne
        \immediate\write\out{\the\cnta}%
        \advance\cntf\@ne
      \fi
    }%
  \immediate\closeout\out
  % print to stdout the count of primes
  \immediate\write128{\the\cntf\space primes were written to file #2}%
}% END OF \SieveOutToFile

\endinput

