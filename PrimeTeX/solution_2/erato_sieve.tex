% Prime Sieve with TeX
% author: jfbu
% date:   2021/07/24
%
% This file defines syntax for a  "Sieve object" in TeX lingua with its
% associated methods: sieve, reset, settocount, outtofile.
%
% Usage:
%
% \NewSieve{<name>}{<range>}
%
% \Sieve <name>.sieve{} executes the sieve
% \Sieve <name>.reset{} resets all array entries to 0
%
% after sieve method has been executed one has access to:
%
% \Sieve <name>.settocount{\macro} sets \macro to the number of primes
% \Sieve <name>.outtofile{<filename>} ships out all found primes,
%    one per line to file <filename>.
%    It will also write to stdout the number of primes found
% 
% Algorithm:
%
% essentially this is the "Base Algorithm" as described in
%
% https://github.com/PlummersSoftwareLLC/Primes/blob/drag-race/CONTRIBUTING.md
%
% In particular:
%
% - inner loops start at factor*factor
%
% - increments in the inner loops go (morally) by step of 2*factor
%
% Some specifics:
%
% - primes are marked as 0, non-primes as 1. The array entries type is
%   a TeX dimension, which pdftex represents (I think) as 32bits words.
%   So more precisely primes are marked as 0sp and non-primes as 1sp.
%   (update: using \dimen's \z@ and \onesp for significantly faster
%    assignments; I knew this would bring some game but I did not
%    expect it as big actually)
%
% - index j stands for the number 2j+1 and starts at j=1. So the sieving
%   array in memory represents only odd numbers, first one 3.
%
% - For a range N (at least 3) we thus instantiate an array of J "32bits
%   words"
%   with maximal J such that 2J+1<=N, i.e. J=(N-1)//2.
%
% Acknowledgements:
%
% - thanks to Kpym for helping me out with docker matters, and
%   for drawing my attention to this "drag race" to start with
%
% TeXnicalities:
%
% - ////                                                   \\\\
%   This library file uses only base Knuth's tex, i.e. it
%   doesn't use the e-TeX extensions such as \numexpr but,
%   only the TeX's syntax with \multiply, \divide, \advance.
%   \\\\                                                   ////
%
% - TeX has no native array type so we use "font dimension arrays".
%   There is no way that I know of to release the memory, only
%   to re-use it (and the array size can not be modified as soon
%   as some other font is defined).
%
% - The native font memory size available in pdftex is only
%   of 8,000,000 words. To sieve up to about or more than
%   about twice this, it is needed to tell pdftex to run
%   with more (static) font memory. For this see file texmf.cnf
%   and/or the README.md
%
% - luatex has dynamic memory allocation so is not constrained
%   but maximal N is at 999,999,999 due to a limitation
%   (which could be lifted) from a shared_batteries.tex
%   auxiliary macro (square root)


% activate @ as a letter for being more mysterious
\catcode`@ 11
% and use also _ as a letter to appear even more TeXnical
\catcode`_ 11

\input shared_batteries.tex

% this will serve to mark composites
\newdimen\onesp \onesp 1sp

% additional count registers 
\newcount\indexa
\newcount\indexb 

\newcount\cnta
\newcount\cntb
\newcount\cntc
\newcount\cntd
\newcount\cstA
\newcount\cstB

\newcount\instance

% we need a write stream for the \Sieve <sieveobject>.outtofile{} method
\newwrite\out

% INSTANTIATION OF A SIEVING OBJECT
% =================================

\def\Sieve #1.{\csname _svobj.#1\endcsname}%

\def\NewSieve#1{%
  % #1 is the name given by user
  % As it may need expansion, we do this expansion once immediately
  \edef\tmp{{#1}}%
  \expandafter\_NewSieve\tmp
}%

\def\_NewSieve#1#2{%
  % This is the "init" method for a new Sieve instantiation
  %
  % #1 is the name given by user to the sieve object.
  % #2 is its range (included in array if odd)
  %
  % Access to its methods or attributes is done either via this syntax:
  %    \Sieve foo.<method>{}
  % where the {} is mandatory,
  %
  % Or via this syntax (with CamelCase for the method name)
  %     \Sieve<method>{foo}
  % (some methods may use a second argument)
  %
  % attributes:
  % - range
  % - instance (used internally)
  %
  % methods:
  % sieve: execute the sieving of the array
  % reset: clear the array, cancelling all sieving work
  % settocount: compute the number of primes in the given range
  %             and assigns it to macro argument
  %     (the sieve method must have been executed first)
  % outtofile: outputs primes one per line to given filename
  %     (the sieve method must have been executed first)
  %
  \expandafter\def\csname _svobj.#1\endcsname##1##{\csname _svobj.#1.##1\endcsname}%
  % range A
  \cstA#2\relax
  %  (the ##1 is only to gobble the {} from the method calling syntax)
  \expandafter\edef\csname _svobj.#1.range\endcsname##1{\the\cstA}%
  %
  % array size 2J+1<= A 
  \cnta\cstA
  \advance\cnta-\@ne
  \divide\cnta\tw@
  %
  % object instance number (used internally to label class instantiations)
  \advance\instance\@ne
  \expandafter\edef\csname _svobj.#1.instance\endcsname##1{\the\instance}%
  %
  % an array of font dimensions (each a 32bits word)
  \cntb 665
  \advance\cntb\instance
  % Initialize a font
  \font\_svobjarray=cmr10 at \cntb sp % each class instantiation via own font
                                % point size 666sp, 667sp, ....
  % Create the array of its parameters 
  \fontdimen\cnta\_svobjarray=\z@
  %
  % Save the name
  \expandafter\let\csname _svobjarray.#1\endcsname\_svobjarray
  %
  % The array entries are set to 0, apart from the first few ones
  % We must fix this here
  \fontdimen1\_svobjarray=\z@ 
  \fontdimen2\_svobjarray=\z@
  \fontdimen3\_svobjarray=\z@
  \fontdimen4\_svobjarray=\z@
  \fontdimen5\_svobjarray=\z@
  \fontdimen6\_svobjarray=\z@
  \fontdimen7\_svobjarray=\z@
  \fontdimen8\_svobjarray=\z@ % needed for usage with luatex
  %
  % sieve method (the ##1 gobbles the {} of the syntax)
  %
  \expandafter\def\csname _svobj.#1.sieve\endcsname##1{%
      \SieveSieve{#1}%
  }%
  %
  % reset method (the ##1 gobbles the {} of the syntax)
  %
  \expandafter\def\csname _svobj.#1.reset\endcsname##1{%
      \SieveReset{#1}%
  }%
  %
  % settocount method (assigns to the second argument the count of primes)
  %
  \expandafter\def\csname _svobj.#1.settocount\endcsname{%
      \SieveSetToCount{#1}%
  }%
  %
  % outtofile method (outputs to the filename given as second argument the list
  % of primes, one per line)
  %
  \expandafter\def\csname _svobj.#1.outtofile\endcsname{%
      \SieveOutToFile{#1}%
  }%
  % 
}% end of \_NewSieve

%
% sieve method
%
\def\SieveSieve#1{%
    % Set \_svobjarray to the font array
    \expandafter\let\expandafter\_svobjarray\csname _svobjarray.#1\endcsname
    % Set \cstA to be the sieve range. Attention that \_svobj.<foo>.range
    % is defined to gobble a {}...
    \cstA\csname _svobj.#1.range\endcsname{}\relax
    % Set \cstB to its square root
    \SetToSqrt\cstA\cstB
    %
    \indexa\z@
    %
    % Start the actual sieving
    % We need to check indices 1..J with 2J+1<=B, so increment J times
    % by 1 the index, initially with value 0.
    \cntb\cstB
    \advance\cntb-\@ne
    \divide\cntb\tw@
    \Replicate{\the\cntb}{%
      \advance\indexa\@ne
      % do we have a prime?
      \ifdim\fontdimen\indexa\_svobjarray=\z@ %
        % yes, so let's sieve out its multiples
        % starting with factor*factor
        \cntc \indexa      % will hold the factor
        \advance\cntc\cntc
        \advance\cntc\@ne % \cntc = 2j+1 = odd factor
        %
        \cntb \cntc
        \multiply \cntb \cntb % factor*factor
        %
        \indexb \cntb
        \divide \indexb \tw@ % (odd number x) //2 gives j such that x=2j+1
        \fontdimen\indexb\_svobjarray=\onesp % factor * factor now sieved out
         %
        \cnta \cstA % Range
        \advance\cnta-\cntb % Range - factor * factor
        \divide \cnta\cntc  % how many odd numbers still to sieve out ?
        \divide \cnta\tw@   % that many
        %
        % let's work with batches of 1000 at a time
        \cntb\cnta
        \divide\cntb\@m % \@m is 1000
        %
        \Replicate{\the\cntb}{%
          \ReplicateM{%
            \advance\indexb\cntc % j steps by factor, thus 2*j+1 by 2*factor
            \fontdimen\indexb\_svobjarray=\onesp % sieved out 
           }%
         }%
        \multiply\cntb\@m
        \advance\cnta-\cntb % this is what remains to be done
        \Replicate{\the\cnta}{%
          \advance\indexb\cntc
          \fontdimen\indexb\_svobjarray=\onesp % sieved out 
        }%
      \fi % end of branch handling sieving out multiples of a prime factor <= sqrt(Range)
    }%
}% end of \SieveSieve

%
% reset method
%
\def\SieveReset#1{%
    % Set \_svobjarray to the font array
    \expandafter\let\expandafter\_svobjarray\csname _svobjarray.#1\endcsname
    % Set \cstA to be the sieve range. Attention that \_svobj.<foo>.range
    % is defined to gobble a {}...
    \cstA\csname _svobj.#1.range\endcsname{}\relax
    %
    \cnta\cstA
    \advance\cnta-\@ne
    \divide\cnta\tw@
    %
    \indexa\z@
    %
        % let's work with batches of 1000 at a time
        \cntb\cnta
        \divide\cntb\@m % \@m is 1000
        %
        \Replicate{\the\cntb}{%
          \ReplicateM{%
            \advance\indexa\@ne
            \fontdimen\indexa\_svobjarray=\z@ %
           }%
         }%
        \multiply\cntb\@m
        \advance\cnta-\cntb % this is what remains to be done
        \Replicate{\the\cnta}{%
          \advance\indexa\@ne
          \fontdimen\indexa\_svobjarray=\z@ %
        }%
}% end of \SieveReset

%
% settocount method
%
\def\SieveSetToCount#1#2{%
    % Set \_svobjarray to the font array
    \expandafter\let\expandafter\_svobjarray\csname _svobjarray.#1\endcsname
    % Set \cstA to be the sieve range. Attention that \_svobj.<foo>.range
    % is defined to gobble a {}...
    \cstA\csname _svobj.#1.range\endcsname{}\relax
    % Get the length of the fontdimen array: 2J+1<=N
    \cnta\cstA
    \advance\cnta-\@ne
    \divide\cnta\tw@
    % index will get incremented by 1, initialize it at 0
    \indexa \z@
    % \cntc will hold the count of primes, set it to 1 as 2 is prime
    \cntc \@ne
    % Proceed by batches of 1000 at a time
    \cntb\cnta
    \divide\cntb\@m
    \Replicate{\the\cntb}{%
        \ReplicateM{%
          \advance\indexa\@ne
          \ifdim\fontdimen\indexa\_svobjarray=\z@
             \advance\cntc\@ne % we have a prime
          \fi
         }%
       }%
    % get how many remain to be checked
    \multiply\cntb\@m
    \advance\cnta-\cntb
    % and do it
    \Replicate{\the\cnta}{%
          \advance\indexa\@ne
          \ifdim\fontdimen\indexa\_svobjarray=\z@
             \advance\cntc\@ne % we have a prime
          \fi
         }%
    % At the end we have incremented \indexa (by 1) \cnta times
    % initial value was 0, final value of index is the array size \cnta
    % Store in macro #2 the result
    \edef#2{\the\cntc}%
}% end of \SieveSetToCount

%
% outtofile method
%
\def\SieveOutToFile#1#2{%
  % #1 is the sieve object name
  % #2 is the file name
  % The sieve method of the #1 object must have been executed first
  \immediate\openout\out=#2\relax
    % Set \_svobjarray to the font array
    \expandafter\let\expandafter\_svobjarray\csname _svobjarray.#1\endcsname
    % Set \cstA to be the sieve range. Attention that \_svobj.<foo>.range
    % is defined to gobble a {}...
    \cstA\csname _svobj.#1.range\endcsname{}\relax
    % Get the length of the fontdimen array: 2J+1<=N
    \cnta\cstA
    \advance\cnta-\@ne
    \divide\cnta\tw@
    % index will get incremented by 1, initialize it at 0
    \indexa \z@
    % 2 is prime
  \immediate\write\out{2}%
  \cntd\@ne
    % Proceed by batches of 1000 at a time
    \cntb\cnta
    \divide\cntb\@m
    \Replicate{\the\cntb}{%
        \ReplicateM{%
          \advance\indexa\@ne
          \ifdim\fontdimen\indexa\_svobjarray=\z@
  % we have a prime
  \cntc\indexa
  \advance\cntc\cntc
  \advance\cntc\@ne
  \immediate\write\out{\the\cntc}%
  \advance\cntd\@ne
          \fi
         }%
      }%
    % get how many remain to be checked
    \multiply\cntb\@m
    \advance\cnta-\cntb
    % and do it
    \Replicate{\the\cnta}{%
          \advance\indexa\@ne
          \ifdim\fontdimen\indexa\_svobjarray=\z@
  % we have a prime
  \cntc\indexa
  \advance\cntc\cntc
  \advance\cntc\@ne
  \immediate\write\out{\the\cntc}%
  \advance\cntd\@ne
          \fi
         }%
    % At the end we have incremented \indexa (by 1) \cnta times
    % initial value was 0, final value of index is the array size \cnta
  \immediate\closeout\out
  \immediate\write128{\the\cntd\space primes were written to file #2}%
}% END OF \SieveOutToFile

\endinput

