% Prime Sieve with TeX
% author: jfbu
% date:   2021/07/24-27
%
% activate @ as a letter for being more mysterious
\catcode`@ 11
% and use also _ as a letter to appear even more TeXnical
\catcode`_ 11

% Load Sieve class

\input erato_sieve.tex

% Parameters for this benchmark run

\def\BenchmarkRange{1000000}%
\def\BenchmarkMinimalDuration{327680}% 5s (327680 = 5*65536)

% Data for validation (will be used only if range is among listed values)
\def\@namedef#1{\expandafter\def\csname #1\endcsname}%
\@namedef{PrimeCount10}{4}%
\@namedef{PrimeCount100}{25}%
\@namedef{PrimeCount1000}{168}%
\@namedef{PrimeCount10000}{1229}%
\@namedef{PrimeCount100000}{9592}%
\@namedef{PrimeCount1000000}{78498}%
\@namedef{PrimeCount10000000}{664579}%
\@namedef{PrimeCount100000000}{5761455}%
\@namedef{PrimeCount999999999}{50847534}% even memory and time aside,
                                % 999999999 is max due to simple-minded
                                % square-root macro in shared_batteries.tex

% luatex compatibility layer (it does not know \pdfresettimer)
\ifdefined\directlua
      % **too** complicated to get this to work with Docker (and I wasted
      % enough time on this)
      % \input pdftexcmds.sty
      % \let\pdfresettimer\pdf@resettimer
      % \let\pdfelapsedtime\pdf@elapsedtime
% https://tex.stackexchange.com/a/32531
\directlua{pdfelapsedtimer_basetime=0}%
\def\pdfresettimer{\directlua{pdfelapsedtimer_basetime = os.clock()}}%
\def\pdfelapsedtime{\numexpr\directlua{tex.print(math.floor((os.clock()-pdfelapsedtimer_basetime)*65536+0.5))}\relax}%
\fi  

% auxiliary for converting \pdfelapsedtime output into seconds
{\catcode`P12\catcode`T12\lowercase{\gdef\strippt#1PT{#1}}}%
\def\SetToElapsedTime#1%
   {\edef#1{\expandafter\strippt\the\dimexpr\pdfelapsedtime sp}}%

% counter for number of passes
\newcount\nbofpasses

% start timing
\pdfresettimer

% \xloop is a loop macro defined in imported shared_batteries.tex
\xloop
    \advance\nbofpasses\@ne
%
% instantiate an object
    \NewSieve{foo\the\nbofpasses}{\BenchmarkRange}%
%
% execute the sieve
    \Sieve foo\the\nbofpasses.sieve{}%
%
% check current duration
\ifnum\pdfelapsedtime<\BenchmarkMinimalDuration\relax
\repeat
%
% freeze elapsed time and convert it to a value in seconds
\SetToElapsedTime\T
%
% compute how many prime numbers were found
\Sieve foo1.settocount{\FooPrimeCount}%
%
% we can not prevent pdftex banner from stdout so write to a tmp file
% which will be then cat by run.sh
\newwrite\out
\immediate\openout\out=\jobname-out.txt
\immediate\write\out{jfbu-tex;\the\nbofpasses;\T;1;algorithm=base,faithful=no,bits=32}
%
% Validate
\expandafter\ifx\csname PrimeCount\BenchmarkRange\endcsname\relax
\else
  \ifnum\csname PrimeCount\BenchmarkRange\endcsname=\FooPrimeCount\relax
  \else
   \immediate\write\out{ERROR IN PRIME COUNT: found \FooPrimeCount, expected
                        \csname PrimeCount\BenchmarkRange\endcsname}%
  \fi
\fi
\bye
