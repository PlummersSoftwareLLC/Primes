% Prime Sieve with Knuth TeX using the 2*3*5  wheel (8 out of 30)
% author: jfbu
% date:   2021/07/27--2021/08/06--2021/08/15
%
% This file defines syntax for a  "Sieve object" in TeX lingua with its
% associated methods: sieve, reset (not implemented), settocount, outtofile.
%
% Usage:
%
% \NewSieve{<name>}{<range>}
%
% **** THE RANGE MUST BE > 5 ****
%
% \Sieve <name>.sieve{} executes the sieve
% %%% \Sieve <name>.reset{} resets all array entries to 0 (not implemented)
%
% after sieve method has been executed one has access to:
%
% \Sieve <name>.settocount{\macro} sets \macro to the number of primes
% \Sieve <name>.outtofile{<filename>} ships out all found primes,
%    one per line to file <filename>.
%    It will also write to stdout the number of primes found
% 
% Algorithm: sieve of Eratosthenes with a 2*3*5 wheel
%
%
% TeXnicalities:
%
% - ////                                                   \\\\
%   This library file uses only base Knuth's tex, i.e. it
%   doesn't use the e-TeX extensions such as \numexpr but,
%   only the TeX's syntax with \multiply, \divide, \advance.
%   \\\\                                                   ////
%
% - TeX has no native array type so we use "font dimension arrays".
%   There is no way that I know of to release the memory, only
%   to re-use it (and the array size can not be modified as soon
%   as some other font is defined).
%
% - Maximal pdftex font memory is at 147,483,647 32bits words
%   and for sieving up to range N, this needs a bit more than
%   N/2 words of memory.
%
% - luatex has dynamic memory allocation so is not constrained
%   but maximal N is at 999,999,999 due to a limitation
%   (which could be lifted) from a shared_batteries.tex
%   auxiliary macro (square root)


% activate @ as a letter for being more mysterious
\catcode`@ 11
% and use also _ as a letter to appear even more TeXnical
\catcode`_ 11

\input shared_batteries.tex

% this will serve to mark composites
\newdimen\onesp \onesp=1sp

% additional count registers 
\newcount\indexa
\newcount\indexb 
\newcount\indexc
\newcount\indexd

\newcount\cntFactor
\newcount\cntFactorClass

\newcount\cnta
\newcount\cntb
\newcount\cntc
\newcount\cntd
\newcount\cnte
\newcount\cntf

\newcount\cstRange
\newcount\cstRangeSqrt

\newcount\instance

% we need a write stream for the \Sieve <sieveobject>.outtofile{} method
\newwrite\out

\newcount\wheelmodulus
% we will work with residue classes prime to 30=2*3*5
\wheelmodulus=30 
% The cogs:
% 1, 7, 11, 13, 17, 19, 23, 29
%
% 
% Some of what follows was originally thought out for the larger
% 2*3*5*7*11=2310 wheel which has 480 cogs.
% We downgrade it to the 8 out of 30 wheel, and we will add
% for optimization usage of some dedicated counts holding
% the successive "deltas" (divided by 2), up to some cyclic
% permutation
% The deltas:
% 3, 2,  1,  2,  1,  2,  3,  1
%
% We will "roll-out" the inner loops by batches
% of length 30 numbers, this would be more complicated
% to do in TeX for the 2310 modulus, in part due to using
% \count registers for which only 255 are available in TeX.
%
% (to handle 480 out of 2310 we could try using the x-->2310-x
%  symmetry, reducing to 240 < 255 cogs,...)
%
% construction of the wheel (it is done once only)
\font\wheelcogids=cmr10 at 666sp
% we will completely ignore even indices
%
% **ultimately** cogids[n] for 1<= n <= 30 will be the number
% of integers m<=n prime to 30. In other terms it is the
% index i of the largest cog value at most equal to n
%
% **the array is configured and meaningful only for odd indices**
%
% 1,3,5,7,9,11,13,15,17,19,21,23,25,27,29,
% | | | | |  |  |  |  |  |  |  |  |  |  | 
% 1,1,1,2,2, 3, 4, 4, 5, 6, 6, 7, 7, 7, 8,
%
%
% reserve enough space:
\fontdimen\wheelmodulus\wheelcogids=\z@
% attention that fontdimen 7 may be set
% (1 will be set directly, 3 and 5 are handled next)
\fontdimen7\wheelcogids=\z@
% sieve out multiples of three up to 30
\indexa=-3
\Replicate{5}{\advance\indexa6 \fontdimen\indexa\wheelcogids=\onesp }%
% sieve out multiples of five up to 30
\indexa=-5
\Replicate{3}{\advance\indexa10 \fontdimen\indexa\wheelcogids=\onesp }%

% before giving to the cogids[n] their final values let's initiate
% the cognext and cogdeltas arrays.

\font\wheelcognext=cmr10 at 667sp
\fontdimen\wheelmodulus\wheelcognext=\z@
% we don't care about the possibly already set values
% we ignore indices not prime to 30
% 1,*,*,7, *,11,13, *,17,19, *,23, *, *,29, 
% |     |     |  |     |  |     |        |
% 7,   11,   13,17,   19,23,   29,       1,

\font\wheelcogdeltas=cmr10 at 668sp
\fontdimen\wheelmodulus\wheelcogdeltas=\z@
% we don't care about the possibly already set values
% we ignore indices not prime to 30
% 1,*,*,7, *,11,13, *,17,19, *,23, *, *,29, 
% |     |     |  |     |  |     |        |
% 3,    2,    1, 2,    1, 2,    3,       1,

% As long as we don't use "\font" we can append
% to the \wheelcogs array.
% Of course it would be more readable code to
% do it explicitly manually, but we do this
% abstractly, as for the lengther 480 of 2310 case
\font\wheelcogs=cmr10 at 669sp
\fontdimen1\wheelcogids=\onesp % 1 is first cog
\fontdimen1\wheelcogs  =\onesp % first cog is at 1
\indexa=\@ne
\indexb=\@ne
\indexc=\@ne
\Replicate{14}{\advance\indexb\tw@
    \ifcase\fontdimen\indexb\wheelcogids % relatively prime to 30
      \advance\indexc\@ne
      \fontdimen\indexc\wheelcogs    =\indexb sp %
      %
      \fontdimen\indexb\wheelcogids  =\indexc sp %
      \fontdimen\indexa\wheelcognext =\indexb sp %
      %
      \cnta=\indexb\advance\cnta-\indexa
%%% main sieving array will use j->2j+1 representation
%%% whereas the cogwheel is represented k->k
%%% so the deltas are divided by 2 here
   \divide\cnta \tw@
%%%
      \fontdimen\indexa\wheelcogdeltas=\cnta sp %
      %
      \indexa=\indexb
    \else
      % attention that wheelcogids array now takes its final entries
      % which are configured only for odd indices, by the way
      \fontdimen\indexb\wheelcogids  =\indexc sp
    \fi
}%
\fontdimen 29\wheelcognext=\onesp %
\fontdimen 29\wheelcogdeltas=\onesp % see divide by 2 above

% we have finished our job with constructing the necessary
% and useful 8 out of 30 wheel data.
% Let's keep memory of the number of relatively prime classes.
\newcount\wheelcogcount
\wheelcogcount=\indexc % i.e. 8

% % debugging during code development to check we got it right
% \indexb-\@ne
%   \immediate\write128{class,
%                       id,
%                       nextclass,
%                       delta (last 2 meaningful only for primes to 30)
%                       }%
% \Replicate{15}{\advance\indexb\tw@
%   \immediate\write128{\the\indexb,
%                       \the\numexpr\fontdimen\indexb\wheelcogids,
%                       \the\numexpr\fontdimen\indexb\wheelcognext,
%                       \the\numexpr\fontdimen\indexb\wheelcogdeltas
%                       }%
% }%
% \immediate\write128{\the\wheelcogcount, \the\numexpr\fontdimen\wheelcogcount\wheelcogs}
% \bye

% INSTANTIATION OF A SIEVING OBJECT
% =================================

\def\Sieve #1.{\csname _svobj.#1\endcsname}%

\def\NewSieve#1{%
  % #1 is the name given by user
  % As it may need expansion, we do this expansion once immediately
  \edef\tmp{{#1}}%
  \expandafter\_NewSieve\tmp
}%

\def\_NewSieve#1#2{%
  % This is the "init" method for a new Sieve instantiation
  %
  % #1 is the name given by user to the sieve object.
  % #2 is its range (included in array if odd)
  %
  % Access to its methods or attributes is done either via this syntax:
  %    \Sieve foo.<method>{}
  % where the {} is mandatory,
  %
  % Or via this syntax (with CamelCase for the method name)
  %     \Sieve<method>{foo}
  % (some methods may use a second argument)
  %
  % attributes:
  % - range
  % - instance (used internally)
  %
  % methods:
  % sieve: execute the sieving of the array
  % reset: clear the array, cancelling all sieving work
  % settocount: compute the number of primes in the given range
  %             and assigns it to macro argument
  %     (the sieve method must have been executed first)
  % outtofile: outputs primes one per line to given filename
  %     (the sieve method must have been executed first)
  %
  \expandafter\def\csname _svobj.#1\endcsname##1##{\csname _svobj.#1.##1\endcsname}%
  % range A
  \cstRange=#2\relax
  %  (the ##1 is only to gobble the {} from the method calling syntax)
  \expandafter\edef\csname _svobj.#1.range\endcsname##1{\the\cstRange}%
  %
  % array size 2J+1<= A
  % (we don't replace A by largest number prime to 30 and less than A
  %  but we possibly should for some -surely- limited gain)
  \cnta=\cstRange
  \advance\cnta-\@ne
  \divide\cnta\tw@
  %
  % object instance number (used internally to label class instantiations)
  \advance\instance\@ne
  \expandafter\edef\csname _svobj.#1.instance\endcsname##1{\the\instance}%
  %
  % an array of font dimensions (each a 32bits word)
  \cntb=669 % as 666, 667, 668, 669 are already in use
  \advance\cntb\instance % instance id at least 1 so this will be 670 at least
  % Initialize a font
  \font\_svobjarray=cmr10 at \cntb sp % each class instantiation impacts memory
  % Create the array of its parameters 
  \fontdimen\cnta\_svobjarray=\z@
  % and save its name in the object attributes
  \expandafter\let\csname _svobjarray.#1\endcsname\_svobjarray
  %
  % Array entries with indices 1 to 7 (or 8 with luatex) are not
  % initialized to zero, let's fix this
  \fontdimen1\_svobjarray=\z@ 
  \fontdimen2\_svobjarray=\z@
  \fontdimen3\_svobjarray=\z@
  \fontdimen4\_svobjarray=\z@
  \fontdimen5\_svobjarray=\z@
  \fontdimen6\_svobjarray=\z@
  \fontdimen7\_svobjarray=\z@
  \fontdimen8\_svobjarray=\z@ % needed for usage with luatex
  %
  % sieve method (the ##1 gobbles the {} of the syntax)
  %
  \expandafter\def\csname _svobj.#1.sieve\endcsname##1{%
      \SieveSieve{#1}%
  }%
  %
  % reset method (the ##1 gobbles the {} of the syntax)
  %
  \expandafter\def\csname _svobj.#1.reset\endcsname##1{%
      \SieveReset{#1}%
  }%
  %
  % settocount method (assigns to the second argument the count of primes)
  %
  \expandafter\def\csname _svobj.#1.settocount\endcsname{%
      \SieveSetToCount{#1}%
  }%
  %
  % outtofile method (outputs to the filename given as second argument the list
  % of primes, one per line)
  %
  \expandafter\def\csname _svobj.#1.outtofile\endcsname{%
      \SieveOutToFile{#1}%
  }%
  % 
}% end of \_NewSieve

%
% sieve method
%
\newcount\cntDeltaA
\newcount\cntDeltaB
\newcount\cntDeltaC
\newcount\cntDeltaD
\newcount\cntDeltaE
\newcount\cntDeltaF
\newcount\cntDeltaG
\newcount\cntDeltaH
\def\SieveSieve#1{%
    % Set \_svobjarray to the font array
    \expandafter\let\expandafter\_svobjarray\csname _svobjarray.#1\endcsname
    % Set \cstRange to be the sieve range. Attention that \_svobj.<foo>.range
    % is defined to gobble a {}...
    \cstRange=\csname _svobj.#1.range\endcsname{}\relax
    % Set \cstRangeSqrt to its square root
    % nota bene: move this to the "init" and add corresponding attribute to
    % object data?
    \SetToSqrt\cstRange\cstRangeSqrt
    %
    % Start the actual wheel sieving
    % 
    % We need to precompute a priori how many elementary steps of the wheel
    % will happen until we have exhausted all potential factors up to B = isqrt(range)
    % 
    \cntb=\cstRangeSqrt
        % make sure we never get an even number modulo 30 as our cogids array indices
        % start at 1 and the array entries are meaningful only for odd indices
        \ifodd\cntb\else\advance\cntb-\@ne\fi
    \cntc=\cntb
    \divide\cntc\wheelmodulus
    \cntd=\cntc
    \multiply\cntd\wheelcogcount
    \multiply\cntc\wheelmodulus
    \advance\cntb-\cntc
    \advance\cntd\fontdimen\cntb\wheelcogids
    %
    % We need to iterate \cntd-1 times to go from
    % 1 to the ultimate N which is prime to 30 and at most equal to B
    % stepping only along numbers all primes to 30
    \advance\cntd-\@ne
% debugging
% \immediate\write128{I will do \the\cntd\space iterations}%
    %
    % \indexa will index sieve array (j maps to 2j+1)
    % and will step by entries of the delta array
    % 
    % \indexc will index cogs modulo 30 (k mapsto k)
    % and will evolve via the entries of the cognext array
    %
    % \cntf will step by one at each iteration and will thus
    % keep information helpful to know for a given factor F
    % exactly how many F*G's will be sieved out
    %
    \indexa=\z@
    \cntFactorClass=\@ne
    \cntf=\@ne
    %
    \Replicate{\the\cntd}{%
      % step indexa by suitable delta
      \advance\indexa\fontdimen\cntFactorClass\wheelcogdeltas
      % and now update the class modulo wheel modulus
      \cntFactorClass=\fontdimen\cntFactorClass\wheelcognext
      % update the iteration count
      \advance\cntf\@ne
      % do we have a prime?
      \ifcase\fontdimen\indexa\_svobjarray
        % yes, so let's sieve out its multiples
        % starting with factor*factor
        %
        \cntFactor=\indexa      % will hold the factor
        \advance\cntFactor\cntFactor
        \advance\cntFactor\@ne  % 2j+1 = odd factor
        %
        \indexb=\cntFactor
        \multiply \indexb \indexb % factor*factor is odd
        %
        \divide \indexb \tw@ 
        \fontdimen\indexb\_svobjarray=\onesp % factor * factor now sieved out
        % Now set up the counts holding deltas for optimizing assignment speed
        % This step is hard-coded for the 8 out of 30 case
            \indexd=\cntFactorClass
        \cntDeltaA=\fontdimen\indexd\wheelcogdeltas
        \multiply\cntDeltaA\cntFactor
            \indexd=\fontdimen\indexd\wheelcognext
        \cntDeltaB=\fontdimen\indexd\wheelcogdeltas
        \multiply\cntDeltaB\cntFactor
            \indexd=\fontdimen\indexd\wheelcognext
        \cntDeltaC=\fontdimen\indexd\wheelcogdeltas
        \multiply\cntDeltaC\cntFactor
            \indexd=\fontdimen\indexd\wheelcognext
        \cntDeltaD=\fontdimen\indexd\wheelcogdeltas
        \multiply\cntDeltaD\cntFactor
            \indexd=\fontdimen\indexd\wheelcognext
        \cntDeltaE=\fontdimen\indexd\wheelcogdeltas
        \multiply\cntDeltaE\cntFactor
            \indexd=\fontdimen\indexd\wheelcognext
        \cntDeltaF=\fontdimen\indexd\wheelcogdeltas
        \multiply\cntDeltaF\cntFactor
            \indexd=\fontdimen\indexd\wheelcognext
        \cntDeltaG=\fontdimen\indexd\wheelcogdeltas
        \multiply\cntDeltaG\cntFactor
            \indexd=\fontdimen\indexd\wheelcognext
        \cntDeltaH=\fontdimen\indexd\wheelcogdeltas
        \multiply\cntDeltaH\cntFactor
        %
        % we now need to know exactly how many iterations are needed
        % 
        \cnta=\cstRange % Range
        \divide\cnta\cntFactor % Range//factor
        % make sure we never get an even number modulo 30 as our cogids array indices
        % start at 1 and the array entries are meaningful only for odd indices
        \ifodd\cnta\else\advance\cnta-\@ne\fi
        %
        \cntb=\cnta
        \cntd=\cnta
        \divide\cnta\wheelmodulus
        \cnte=\cnta
        \multiply\cnta\wheelmodulus
        \advance\cntd-\cnta % this is modulo modulus
        %
        \multiply\cnte\wheelcogcount
        \advance\cnte\fontdimen\cntd\wheelcogids % get id of maximal cog to
                                % the left
        % get finally how many iterations must be done: cnte - cntf
        \advance\cnte-\cntf
        % cofactor G is initially F which maps to \cntFactorClass modulo 30
        % indexb will be (F * G - 1)/2, currently G = F
        \cnta=\cnte
        \divide\cnta\@m % \@m is 1000
        %
        \Replicate{\the\cnta}{%
          \ReplicateOneTwoFive{% 1000 = 125 x 8
            % A
            % step indexb by suitable delta * factor
            \advance\indexb\cntDeltaA
            % and mark this index as sieved out
            \fontdimen\indexb\_svobjarray=\onesp % sieved out
            % B
            \advance\indexb\cntDeltaB
            % and mark this index as sieved out
            \fontdimen\indexb\_svobjarray=\onesp % sieved out 
            % C
            \advance\indexb\cntDeltaC
            % and mark this index as sieved out
            \fontdimen\indexb\_svobjarray=\onesp % sieved out 
            % D
            \advance\indexb\cntDeltaD
            % and mark this index as sieved out
            \fontdimen\indexb\_svobjarray=\onesp % sieved out 
            % E
            \advance\indexb\cntDeltaE
            % and mark this index as sieved out
            \fontdimen\indexb\_svobjarray=\onesp % sieved out 
            % F
            \advance\indexb\cntDeltaF
            % and mark this index as sieved out
            \fontdimen\indexb\_svobjarray=\onesp % sieved out 
            % G
            \advance\indexb\cntDeltaG
            % and mark this index as sieved out
            \fontdimen\indexb\_svobjarray=\onesp % sieved out 
            % H
            \advance\indexb\cntDeltaH
            % and mark this index as sieved out
            \fontdimen\indexb\_svobjarray=\onesp % sieved out 
           }%
         }%
        \multiply\cnta\@m
        \advance\cnte-\cnta % this is what remains to be done (< 1000)
        \cnta\cnte
        \divide\cnta\wheelcogcount
        \Replicate{\the\cnta}{%
            % A step indexb by suitable delta * factor
            \advance\indexb\cntDeltaA
            % and mark this index as sieved out
            \fontdimen\indexb\_svobjarray=\onesp % sieved out
            % B
            \advance\indexb\cntDeltaB
            % and mark this index as sieved out
            \fontdimen\indexb\_svobjarray=\onesp % sieved out 
            % C
            \advance\indexb\cntDeltaC
            % and mark this index as sieved out
            \fontdimen\indexb\_svobjarray=\onesp % sieved out 
            % D
            \advance\indexb\cntDeltaD
            % and mark this index as sieved out
            \fontdimen\indexb\_svobjarray=\onesp % sieved out 
            % E
            \advance\indexb\cntDeltaE
            % and mark this index as sieved out
            \fontdimen\indexb\_svobjarray=\onesp % sieved out 
            % F
            \advance\indexb\cntDeltaF
            % and mark this index as sieved out
            \fontdimen\indexb\_svobjarray=\onesp % sieved out 
            % G
            \advance\indexb\cntDeltaG
            % and mark this index as sieved out
            \fontdimen\indexb\_svobjarray=\onesp % sieved out 
            % H
            \advance\indexb\cntDeltaH
            % and mark this index as sieved out
            \fontdimen\indexb\_svobjarray=\onesp % sieved out 
        }%
        \multiply\cnta\wheelcogcount
        \advance\cnte-\cnta % this is what remains to be done
        % \cntFactorClass is also same class modulo 30 as class of current coFactor
        \indexd=\cntFactorClass
        % for the last 0 to 7 needed steps, let's use slower technique
        \Replicate{\the\cnte}{%
            % step indexb by suitable delta * factor
            \cntd=\cntFactor % factor
            \multiply\cntd\fontdimen\indexd\wheelcogdeltas
            \advance\indexb\cntd
            % and mark this index as sieved out
            \fontdimen\indexb\_svobjarray=\onesp % sieved out 
            % update the class of G modulo modulus
            \indexd=\fontdimen\indexd\wheelcognext
        }%
      \fi
    }%
}% end of \SieveSieve


%
% settocount method
%
\def\SieveSetToCount#1#2{%
  % #1 is the sieve object name
  % #2 is the macro to define
  % The sieve method of the #1 object must have been executed first
    % Set \_svobjarray to the font array
    \expandafter\let\expandafter\_svobjarray\csname _svobjarray.#1\endcsname
    % Set \cstRange to be the sieve range. Attention that \_svobj.<foo>.range
    % is defined to gobble a {}...
    \cstRange=\csname _svobj.#1.range\endcsname{}\relax
    %
    % We need to precompute a priori how many elementary steps of the wheel
    % will happen until we have exhausted all potential factors up to B = isqrt(range)
    % 
    \cnta=\cstRange
        % make sure we never get an even number modulo 30 as our cogids indices
        % start at 1 and the array entries are meaningful only for odd indices
        \ifodd\cnta\else\advance\cnta-\@ne\fi
    \cntc=\cnta
    \divide\cntc\wheelmodulus
    \cntd=\cntc
    \multiply\cntd\wheelcogcount
    \multiply\cntc\wheelmodulus
    \advance\cnta-\cntc % this is A modulo 30 (A odd so this is odd too)
    \advance\cntd\fontdimen\cnta\wheelcogids
    %
    % We need to iterate \cntd-1 times to go from
    % 1 to the maximal N which is prime to 30 and at most A
    % stepping along the way only through numbers primes to 30
    \advance\cntd-\@ne
    %
    % \indexa will index sieve array (j maps to 2j+1)
    % and will step by entries of the delta array
    % 
    % \indexc will index cogs modulo 30 (k mapsto k)
    % and will evolve via the entries of the cognext array
    %
    % \cntf will stepped by one each time a prime is found
    %
    \indexa=\z@
    \indexc=\@ne
    \cntf=3 % 2, 3, 5
    %
        % This step is hard-coded for the 8 out of 30 case
        % but we could be even more explicit via 3, 2,  1,  2,  1,  2,  3,  1
        % leaving this way for easier generalization to 480 out of 2310 case
    \indexd=\indexc % 1
        \cntDeltaA=\fontdimen\indexd\wheelcogdeltas
            \indexd=\fontdimen\indexd\wheelcognext % 7
        \cntDeltaB=\fontdimen\indexd\wheelcogdeltas
            \indexd=\fontdimen\indexd\wheelcognext % 11
        \cntDeltaC=\fontdimen\indexd\wheelcogdeltas
            \indexd=\fontdimen\indexd\wheelcognext % 13
        \cntDeltaD=\fontdimen\indexd\wheelcogdeltas
            \indexd=\fontdimen\indexd\wheelcognext % 17
        \cntDeltaE=\fontdimen\indexd\wheelcogdeltas
            \indexd=\fontdimen\indexd\wheelcognext % 19
        \cntDeltaF=\fontdimen\indexd\wheelcogdeltas
            \indexd=\fontdimen\indexd\wheelcognext % 23
        \cntDeltaG=\fontdimen\indexd\wheelcogdeltas
            \indexd=\fontdimen\indexd\wheelcognext % 29
        \cntDeltaH=\fontdimen\indexd\wheelcogdeltas
    % let's proceed by batches of 1000 at a time
    \cntb=\cntd
    \divide\cntb\@m % divide by 1000
    \Replicate{\the\cntb}{%
     \ReplicateOneTwoFive{%
      % step indexa by suitable delta
      % A
      \advance\indexa\cntDeltaA
      % do we have a prime?
      \ifcase\fontdimen\indexa\_svobjarray
        % we have a prime
        \advance\cntf\@ne
      \fi
      % B
      \advance\indexa\cntDeltaB
      % do we have a prime?
      \ifcase\fontdimen\indexa\_svobjarray
        % we have a prime
        \advance\cntf\@ne
      \fi
      % C
      \advance\indexa\cntDeltaC
      % do we have a prime?
      \ifcase\fontdimen\indexa\_svobjarray
        % we have a prime
        \advance\cntf\@ne
      \fi
      % D
      \advance\indexa\cntDeltaD
      % do we have a prime?
      \ifcase\fontdimen\indexa\_svobjarray
        % we have a prime
        \advance\cntf\@ne
      \fi
      % E
      \advance\indexa\cntDeltaE
      % do we have a prime?
      \ifcase\fontdimen\indexa\_svobjarray
        % we have a prime
        \advance\cntf\@ne
      \fi
      % F
      \advance\indexa\cntDeltaF
      % do we have a prime?
      \ifcase\fontdimen\indexa\_svobjarray
        % we have a prime
        \advance\cntf\@ne
      \fi
      % G
      \advance\indexa\cntDeltaG
      % do we have a prime?
      \ifcase\fontdimen\indexa\_svobjarray
        % we have a prime
        \advance\cntf\@ne
      \fi
      % H
      \advance\indexa\cntDeltaH
      % do we have a prime?
      \ifcase\fontdimen\indexa\_svobjarray
        % we have a prime
        \advance\cntf\@ne
      \fi
     }%
    }%
    % get how many remain to be checked
    \multiply\cntb\@m
    \advance\cntd-\cntb % this is what remains to be done (< 1000)
    \cntb\cntd
    % and do it
    \divide\cntb\wheelcogcount
    \Replicate{\the\cntb}{%
      % step indexa by suitable delta
      % A
      \advance\indexa\cntDeltaA
      % do we have a prime?
      \ifcase\fontdimen\indexa\_svobjarray
        % we have a prime
        \advance\cntf\@ne
      \fi
      % B
      \advance\indexa\cntDeltaB
      % do we have a prime?
      \ifcase\fontdimen\indexa\_svobjarray
        % we have a prime
        \advance\cntf\@ne
      \fi
      % C
      \advance\indexa\cntDeltaC
      % do we have a prime?
      \ifcase\fontdimen\indexa\_svobjarray
        % we have a prime
        \advance\cntf\@ne
      \fi
      % D
      \advance\indexa\cntDeltaD
      % do we have a prime?
      \ifcase\fontdimen\indexa\_svobjarray
        % we have a prime
        \advance\cntf\@ne
      \fi
      % E
      \advance\indexa\cntDeltaE
      % do we have a prime?
      \ifcase\fontdimen\indexa\_svobjarray
        % we have a prime
        \advance\cntf\@ne
      \fi
      % F
      \advance\indexa\cntDeltaF
      % do we have a prime?
      \ifcase\fontdimen\indexa\_svobjarray
        % we have a prime
        \advance\cntf\@ne
      \fi
      % G
      \advance\indexa\cntDeltaG
      % do we have a prime?
      \ifcase\fontdimen\indexa\_svobjarray
        % we have a prime
        \advance\cntf\@ne
      \fi
      % H
      \advance\indexa\cntDeltaH
      % do we have a prime?
      \ifcase\fontdimen\indexa\_svobjarray
        % we have a prime
        \advance\cntf\@ne
      \fi
    }%
    \multiply\cntb\wheelcogcount
    \advance\cntd-\cntb
    % \indexc equal to 1 modulo 30 is valid unchanged
    \Replicate{\the\cntd}{%
      % step indexa by suitable delta
      \advance\indexa\fontdimen\indexc\wheelcogdeltas
      % and now update the class modulo wheel modulus
      \indexc=\fontdimen\indexc\wheelcognext
      % do we have a prime?
      \ifcase\fontdimen\indexa\_svobjarray
      % we have a prime
        \advance\cntf\@ne
      \fi
    }%
  % set macro #2 to count of primes
  \edef#2{\the\cntf}%
}% END OF \SieveSetToCount

%
% outtofile method. Writing out to a file is significantly
% more time costly than doing the actual sieving. I tried
% with no great success to group the primes into batches
% before each \write.
%
\def\SieveOutToFile#1#2{%
  % #1 is the sieve object name
  % #2 is the file name
  % The sieve method of the #1 object must have been executed first
  \immediate\openout\out=#2\relax
  \immediate\write\out{2}%
  \immediate\write\out{3}%
  \immediate\write\out{5}%
    % Set \_svobjarray to the font array
    \expandafter\let\expandafter\_svobjarray\csname _svobjarray.#1\endcsname
    % Set \cstRange to be the sieve range. Attention that \_svobj.<foo>.range
    % is defined to gobble a {}...
    \cstRange=\csname _svobj.#1.range\endcsname{}\relax
    %
    % We need to precompute a priori how many elementary steps of the wheel
    % will happen until we have exhausted all potential factors up to B = isqrt(range)
    % 
    \cnta=\cstRange
        % make sure we never get an even number modulo 30 as our cogids indices
        % start at 1 and the array entries are meaningful only for odd indices
        \ifodd\cnta\else\advance\cnta-\@ne\fi
    \cntc=\cnta
    \divide\cntc\wheelmodulus
    \cntd=\cntc
    \multiply\cntd\wheelcogcount
    \multiply\cntc\wheelmodulus
    \advance\cnta-\cntc % this is A modulo 30 (A odd so this is odd too)
    \advance\cntd\fontdimen\cnta\wheelcogids
    %
    % We need to iterate \cntd-1 times to go from
    % 1 to the maximal N which is prime to 30 and at most A
    % stepping along the way only through numbers primes to 30
    \advance\cntd-\@ne
% debugging
% \immediate\write128{I will do \the\cntd\space iterations}%
    %
    % \indexa will index sieve array (j maps to 2j+1)
    % and will step by entries of the delta array
    % 
    % \indexc will index cogs modulo 30 (k mapsto k)
    % and will evolve via the entries of the cognext array
    %
    % \cntf will stepped by one each time a prime is found
    %
    \indexa=\z@
    \indexc=\@ne
    \cntf=3 % 2, 3, 5
    %
        % This step is hard-coded for the 8 out of 30 case
        % but we could be even more explicit via 3, 2,  1,  2,  1,  2,  3,  1
        % leaving this way for easier generalization to 480 out of 2310 case
    \indexd=\indexc % 1
        \cntDeltaA=\fontdimen\indexd\wheelcogdeltas
            \indexd=\fontdimen\indexd\wheelcognext % 7
        \cntDeltaB=\fontdimen\indexd\wheelcogdeltas
            \indexd=\fontdimen\indexd\wheelcognext % 11
        \cntDeltaC=\fontdimen\indexd\wheelcogdeltas
            \indexd=\fontdimen\indexd\wheelcognext % 13
        \cntDeltaD=\fontdimen\indexd\wheelcogdeltas
            \indexd=\fontdimen\indexd\wheelcognext % 17
        \cntDeltaE=\fontdimen\indexd\wheelcogdeltas
            \indexd=\fontdimen\indexd\wheelcognext % 19
        \cntDeltaF=\fontdimen\indexd\wheelcogdeltas
            \indexd=\fontdimen\indexd\wheelcognext % 23
        \cntDeltaG=\fontdimen\indexd\wheelcogdeltas
            \indexd=\fontdimen\indexd\wheelcognext % 29
        \cntDeltaH=\fontdimen\indexd\wheelcogdeltas
    % let's proceed by batches if 1000 at a time
    \cntb=\cntd
    \divide\cntb\@m % divide by 1000
    \Replicate{\the\cntb}{%
     \ReplicateOneTwoFive{%
      % step indexa by suitable delta
      % A
      \advance\indexa\cntDeltaA
      % do we have a prime?
      \ifcase\fontdimen\indexa\_svobjarray
        % we have a prime
        \cnta=\indexa
        \advance\cnta\cnta
        \advance\cnta\@ne
        \immediate\write\out{\the\cnta}%
        \advance\cntf\@ne
      \fi
      % B
      \advance\indexa\cntDeltaB
      % do we have a prime?
      \ifcase\fontdimen\indexa\_svobjarray
        % we have a prime
        \cnta=\indexa
        \advance\cnta\cnta
        \advance\cnta\@ne
        \immediate\write\out{\the\cnta}%
        \advance\cntf\@ne
      \fi
      % C
      \advance\indexa\cntDeltaC
      % do we have a prime?
      \ifcase\fontdimen\indexa\_svobjarray
        % we have a prime
        \cnta=\indexa
        \advance\cnta\cnta
        \advance\cnta\@ne
        \immediate\write\out{\the\cnta}%
        \advance\cntf\@ne
      \fi
      % D
      \advance\indexa\cntDeltaD
      % do we have a prime?
      \ifcase\fontdimen\indexa\_svobjarray
        % we have a prime
        \cnta=\indexa
        \advance\cnta\cnta
        \advance\cnta\@ne
        \immediate\write\out{\the\cnta}%
        \advance\cntf\@ne
      \fi
      % E
      \advance\indexa\cntDeltaE
      % do we have a prime?
      \ifcase\fontdimen\indexa\_svobjarray
        % we have a prime
        \cnta=\indexa
        \advance\cnta\cnta
        \advance\cnta\@ne
        \immediate\write\out{\the\cnta}%
        \advance\cntf\@ne
      \fi
      % F
      \advance\indexa\cntDeltaF
      % do we have a prime?
      \ifcase\fontdimen\indexa\_svobjarray
        % we have a prime
        \cnta=\indexa
        \advance\cnta\cnta
        \advance\cnta\@ne
        \immediate\write\out{\the\cnta}%
        \advance\cntf\@ne
      \fi
      % G
      \advance\indexa\cntDeltaG
      % do we have a prime?
      \ifcase\fontdimen\indexa\_svobjarray
        % we have a prime
        \cnta=\indexa
        \advance\cnta\cnta
        \advance\cnta\@ne
        \immediate\write\out{\the\cnta}%
        \advance\cntf\@ne
      \fi
      % H
      \advance\indexa\cntDeltaH
      % do we have a prime?
      \ifcase\fontdimen\indexa\_svobjarray
        % we have a prime
        \cnta=\indexa
        \advance\cnta\cnta
        \advance\cnta\@ne
        \immediate\write\out{\the\cnta}%
        \advance\cntf\@ne
      \fi
     }%
    }%
    % get how many remain to be checked
    \multiply\cntb\@m
    \advance\cntd-\cntb % this is what remains to be done (< 1000)
    \cntb\cntd
    % and do it
    \divide\cntb\wheelcogcount
    \Replicate{\the\cntb}{%
      % step indexa by suitable delta
      % A
      \advance\indexa\cntDeltaA
      % do we have a prime?
      \ifcase\fontdimen\indexa\_svobjarray
        % we have a prime
        \cnta=\indexa
        \advance\cnta\cnta
        \advance\cnta\@ne
        \immediate\write\out{\the\cnta}%
        \advance\cntf\@ne
      \fi
      % B
      \advance\indexa\cntDeltaB
      % do we have a prime?
      \ifcase\fontdimen\indexa\_svobjarray
        % we have a prime
        \cnta=\indexa
        \advance\cnta\cnta
        \advance\cnta\@ne
        \immediate\write\out{\the\cnta}%
        \advance\cntf\@ne
      \fi
      % C
      \advance\indexa\cntDeltaC
      % do we have a prime?
      \ifcase\fontdimen\indexa\_svobjarray
        % we have a prime
        \cnta=\indexa
        \advance\cnta\cnta
        \advance\cnta\@ne
        \immediate\write\out{\the\cnta}%
        \advance\cntf\@ne
      \fi
      % D
      \advance\indexa\cntDeltaD
      % do we have a prime?
      \ifcase\fontdimen\indexa\_svobjarray
        % we have a prime
        \cnta=\indexa
        \advance\cnta\cnta
        \advance\cnta\@ne
        \immediate\write\out{\the\cnta}%
        \advance\cntf\@ne
      \fi
      % E
      \advance\indexa\cntDeltaE
      % do we have a prime?
      \ifcase\fontdimen\indexa\_svobjarray
        % we have a prime
        \cnta=\indexa
        \advance\cnta\cnta
        \advance\cnta\@ne
        \immediate\write\out{\the\cnta}%
        \advance\cntf\@ne
      \fi
      % F
      \advance\indexa\cntDeltaF
      % do we have a prime?
      \ifcase\fontdimen\indexa\_svobjarray
        % we have a prime
        \cnta=\indexa
        \advance\cnta\cnta
        \advance\cnta\@ne
        \immediate\write\out{\the\cnta}%
        \advance\cntf\@ne
      \fi
      % G
      \advance\indexa\cntDeltaG
      % do we have a prime?
      \ifcase\fontdimen\indexa\_svobjarray
        % we have a prime
        \cnta=\indexa
        \advance\cnta\cnta
        \advance\cnta\@ne
        \immediate\write\out{\the\cnta}%
        \advance\cntf\@ne
      \fi
      % H
      \advance\indexa\cntDeltaH
      % do we have a prime?
      \ifcase\fontdimen\indexa\_svobjarray
        % we have a prime
        \cnta=\indexa
        \advance\cnta\cnta
        \advance\cnta\@ne
        \immediate\write\out{\the\cnta}%
        \advance\cntf\@ne
      \fi
    }%
    \multiply\cntb\wheelcogcount
    \advance\cntd-\cntb
    % \indexc equal to 1 modulo 30 is valid unchanged
    \Replicate{\the\cntd}{%
      % step indexa by suitable delta
      \advance\indexa\fontdimen\indexc\wheelcogdeltas
      % and now update the class modulo wheel modulus
      \indexc=\fontdimen\indexc\wheelcognext
      % do we have a prime?
      \ifcase\fontdimen\indexa\_svobjarray
      % we have a prime
        \cnta=\indexa
        \advance\cnta\cnta
        \advance\cnta\@ne
        \immediate\write\out{\the\cnta}%
        \advance\cntf\@ne
      \fi
    }%
  \immediate\closeout\out
  % print to stdout the count of primes
  \immediate\write128{\the\cntf\space primes were written to file #2}%
}% END OF \SieveOutToFile

\endinput

